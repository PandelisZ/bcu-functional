\section{Essay}
How are Functional Programming techniques currently related to web development
 and what connections does it have to the development of web services?

\bigskip

There are many requirements that web services and applications need to fulfill
which often leads to sites becoming very hard to maintain in the long term. As
features get added and new users begin to use the services developed the
project can become enormously monolithic.

Within the last decade web developers are trying to solve these challenges
of constantly developing new features but avoiding a monolithic code base
by splitting their services into \textit{micro-services}. These are smaller
more manageable chunks of an overall web service or product which can
exist on their own without any dependency on the rest of the codebase.

This new shift in the way web developers are working and creating services
means they can use any toolset or language they desire to build out a micro-service.
This has allowed web developers to try out new methods and techniques integrating
more functional and declarative techniques into the services they are producing.

Functional languages and the techniques used also make it significantly easier
to secure a web service and maintain high uptime. This is because each
component of the code only ever needs to mutate requests and data. There are less
complex data structures and objects to pass around. \parencite{da2013comparing}

This high degree of performance and reliability is what helps modern
web business such as Facebook maintain
the enormous amount of data throughput they achieve in their datacenter. Most
of their content is served from Haskell backed servers. \parencite{fbconf}

There has also been a shift in the way web developers think about and develop
new services. They are increasingly using more and more declarative syntax making small
but highly reusable functions that can be picks out of and moved from project to project.
This is a side effect of the scale of some engineering teams needing to maintain
the readability and maintainability of their codebase. This process of defining
reusable components makes designing and writing new features extremely easy and
quick as most of the work is already done and validated.

Furthermore there is a trend towards cloud computing which is shaking up the way
services interact with each other and process data internally.
This is leading to a shift towards highly distributed workloads which traditional
OOP based languages and paradigms cant keep up with. We are begging to see functional
languages excel in this space because of their highly parallelized nature.\parencite{chechina2012design}

To conclude there is a culmination of factors that have been adding up over the years
in conjunction with many new younger programmers introducing new techniques and
tools to the web development ecosystem, that is leading to more functional and declarative
systems. More developers are choosing to work with functional for reliability, security
and scalability reasons. 
