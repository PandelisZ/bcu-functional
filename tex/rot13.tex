\section{Rot13}

\subsection{Brief}

Implement a rot-13 utility program (like tr, which acts like a common UNIX utility, performing a line-by-line rot-13 encoding of every line of input contained in each file listed on its command line.

\subsection{Planning and Design}

ROT13 is a type of Caesar cypher in which each character is replaced with the corresponding letter 13 spaces down the alphabet. When the cypher gets to the end of the usable alphabet it looks back around to A.

\subsubsection{Defining behaviour}

As we aim to build a command line utility program we will be creating a simple
argument based console utility. We will be defining a single entry point to the
ROT13 utility in which we will accept a single argument. This argument will be
the relative or absolute file path to a file on the users disc.

We may also want to implement a method for accepting a bigger input to the
utility in order to be able to pipe a file into it. Such that you could use
the cat utility to pipe the contents of a file into ROT13.

\bigskip

\begin{lstlisting}[caption=Examples of how a user will interact with the utility from the console]
> rot13 file.txt
> rot13 /user/home/u/file.txt
> cat file.txt | rot13
\end{lstlisting}


\subsubsection{Approach}

As we are essentially dealing with a type of encryption we will also face the
difficulty of verifying that our implementation works and is reliable. Due to
the expected and highly reproducible behaviour of encryption and cyphers the same
input will always have the same output.

In order to make sure our code is reliable and is doing exactly what we expect
it to do during all phases of development we will be using a test driven
development process. This means at every stage we will be defining a test for
whatever function or feature we are writing at the time. That feature or function
will not be deemed as working until it passes its test. When it does it will be
deemed reliable enough to use repeatedly throughout the rest of the application.

This will allow us to build up and reuse our functions piece by piece and quickly
fix any errors with the functions we work on later on because we will have already
tested the functions feeding into the later functions earlier.

In order to keep the code style as declarative as possible, we will only write
unary functions that can be easily pipelined. As we build up our application
we will add to the pipeline eventually creating the full feature set of operations.

This will also aid in our testing as we can test minimal code and write many more
tests.

\subsubsection{Technical Requirements}

In order to effectively define and execute tests we will be needed a testing
framework. I have chosen to use NUnit and all test will be defined within the
NUnit framework.

In order to keep our code clean and reusable the project will be split into 4
files and modules. The files will be (in execution order):
\begin{enumerate}
  \item File.fs - File IO operations
  \item Rot.fs - ROT encryption and transformation
  \item Tests.fs - Unit tests
  \item Main.fs - Entry point where File meets Rot function
\end{enumerate}

\subsection{Implementation}

The full source code is available in the appendix. Please note that unit tests will be shown inline
with their accompanying function where applicable. In reality all tests exist in Tests.fs

Please also note that double single quotes ('') are used in test function names
as there were problems displaying them as double back ticks ( \` )

\subsubsection{Rot.fs}
Rot.fs is the main logic of our utility. This is where the shifting and
cyphering will take place. This file contains the following modules and functions.

\begin{lstlisting}[language=F]
  module Rot
  module Alphabet
      explode: string -> char list
      implode: char list -> string
  module Encrypt
      isUpper: char -> bool
      toLower: char -> char
      toUpper: char -> char
      rec shift: char -> char
      rotify: string -> string
\end{lstlisting}

I have split Rot down into two submodules. Rot.Alphabet and Rot.Encrypt. Rot.Alphabet
handles transforming the incoming strings to the appropriate char Lists while
Rot.Encrypt carries out the cyphering of the incoming text, utilising Rot.Alphabet

\bigskip

\textbf{module Alphabet}

\begin{lstlisting}[language=F]
let explode (s:string) =
        [for char in s -> char]

[<Test>]
let ''Expand a string into a list with each character as an element''() =
      let i = "Hello"
      let list = ['H';'e';'l';'l';'o']
      Assert.AreEqual(list, Rot.Alphabet.explode i)
\end{lstlisting}

Explode takes a string argument and will expand it out into a List of Chars.
For every \textbf{char} found in the string \textbf{s} it will yield the
\textbf{char} at that position.

\begin{lstlisting}[language=F]
let implode (xs:char list) =
      let sb = System.Text.StringBuilder(xs.Length)
      xs |> List.iter (sb.Append >> ignore)
      sb.ToString()

[<Test>]
let ''Implode the created list into a string containing the characters in the list''() =
      let list = ['H';'e';'l';'l';'o']
      Assert.AreEqual("Hello", Rot.Alphabet.implode list)
\end{lstlisting}
Implode as the name suggests does the opposite of explode. It takes a List of Chars
and returns back a string of those chars. Unfortunately I couldn't not find a way
around piping the output of sb.Append to ignore so this function is not as declarative as it could be.

\textbf{module Encrypt}

\begin{lstlisting}[language=F]
let isUpper (c:char) =
    System.Char.IsUpper c

let toLower (c:char) =
    System.Char.ToLower c

let toUpper (c:char) =
    System.Char.ToUpper c
\end{lstlisting}
These functions exists purely as syntactic sugar to make it easier to use them
in the next function call.

\begin{lstlisting}[language=F]
let rec shift (c:char) =
     if(System.Char.IsLetter c) then
         let num = int(c) - int('A')
         let offsetNum = (num+13)%26
         if(isUpper c) then
             char(offsetNum + int('A'))
         else
             toUpper(c)
             |> shift
             |> toLower
     else
         c
[<Test>]
let ''Handle non text shifting''() =
   let i = '@'
   let o = '@'
   Assert.AreEqual(o, Rot.Encrypt.shift(i))

[<Test>]
let ''Shift 13 characters allong alphabet''() =
   let i = 'A'
   let o = 'N'
   Assert.AreEqual(o, Rot.Encrypt.shift(i))

[<Test>]
let ''Shift 13 characters allong alphabet lowercase''() =
   let i = 'a'
   let o = 'n'
   Assert.AreEqual(o, Rot.Encrypt.shift(i))

\end{lstlisting}
Shift is what takes care of the cyphering process. It accepts a character
and will then check to see if the character is uppercase. If it is uppercase
it will get the raw character value and then subtract it from the character value
for A.

In order to wrap around the alphabet our offsetNum will be a the remainder of the length of the alphabet, 26.
The resulting character code is offsetNum + our starting character A. It then returns the appropriate character
for the value

In order to make our logic work without writing too much extra code I chose to pass the lowercase case letter
back through shift recursively as uppercase and then back to lowercase. This is an area that could be improved on.

There is also a check to see if the character is in fact a letter. If it is not a letter we cannot cypher it so we pass it back untouched.

\begin{lstlisting}[language=F]
let rotify(str:string) =
    Alphabet.explode(str)
    |> List.map (fun c -> shift(c))
    |> Alphabet.implode

[<Test>]
let ''Rot13 this string''() =
    let i = "ABCDEFGHIJKLMNOPQRSTUVWXYZabcdefghijklmnopqrstuvwxyz"
    let o = "NOPQRSTUVWXYZABCDEFGHIJKLMnopqrstuvwxyzabcdefghijklm"
    Assert.AreEqual(o, Rot.Encrypt.rotify(i))

[<Test>]
let ''Handle non text inputs''() =
    let i = "ABCD!FGHI-KLMNO4QRSTUV6XYZabcdefghijk-mnopqrs@uvwxyz"
    let o = "NOPQ!STUV-XYZAB4DEFGHI6KLMnopqrstuvwx-zabcdef@hijklm"
    Assert.AreEqual(o, Rot.Encrypt.rotify(i))
\end{lstlisting}

The rotify function takes all our independent function logic and creates a pipeline.
We take in a string and explode it into a list of characters. We then pipe that list
into a map function which applies the shift on each character. Finally we return back
a string of the rotified text.

\subsubsection{File.fs}

File.fs handles our incoming file from the system. We want to do some checks to make
sure the file actually exists before we start passing it to our rotify functions so we handle that here.

\begin{lstlisting}[language=F]
module FileIO
    Open: string -> string list
\end{lstlisting}

\textbf{module FileIO}
\begin{lstlisting}[language=F]
  let Open(path:string) =
      match File.Exists(path) with
      | true ->
          File.ReadAllLines(path)
          |> Array.toList
      | _ -> invalidArg "file" "The file specified does not exist"
\end{lstlisting}
When supplying the utility with a file argument we want to check that the
file actually exists. We do a pattern match on File.Exists() and if it is true
we can continue and read out all the lines to an Array.

If not we return an error

\subsubsection{Main.fs}
\begin{lstlisting}[language=F]
namespace Rot13
module Main
    entry: string[] -> int
\end{lstlisting}

\textbf{module Main}
\begin{lstlisting}[language=F]
[<EntryPoint>]
let entry argv =
  if (argv.Length = 0) then
        printfn("Please provide the route to a file on the system to process")
        1 //No file found so return an error exit code
  else
    FileIO.Open argv.[0]
    |> List.map(fun line -> Rot.Encrypt.rotify(line) |> printfn("%s") )
    |> ignore
  0
\end{lstlisting}

This is where our utility entry point exists. Before we even attempt to read a file
we need to check that the user has given us a path to look in first. If not
we will let the user know via a printfn and exit with an error code of 1.

If we do have a path we can attempt to open the file. Once opened FileIO gives us an
Array of lines. We can map over that array and rotify each of them and subsequently print them to stdout.
We don't expect an output from printfn so we have to send it to ignore.

The utility then exits with a code 0.

\subsection{Solution Analysis}
